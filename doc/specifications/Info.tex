\documentclass[10pt,a4paper]{article}

%% Formateo del título del documento
\title{\Huge C$+$$-$}
\author{Juan Diego Barrado Daganzo, Javier Saras González y Daniel González Arbelo \\ 4º de Carrera}
\date{\today\footnote{Este documento se actualiza, para consultar las últimas versiones entrar en el enlace \url{https://github.com/JuanDiegoBarrado/PracticaPL}}}

%% Formateo del estilo de escritura y de la pagina
\pagestyle{plain}               % Estilo de página
\setlength{\parskip}{0.35cm}    % Edicion de espaciado
\setlength{\parindent}{0cm}     % Edicion de sangría
\clubpenalty=10000              % Llíneas viudas NO
\widowpenalty=10000             % Líneas viudas NO

%% Para establecer las medidas de los margenes
\usepackage[top=2.5cm, bottom=2.5cm, left=3cm, right=3cm]{geometry} 
%% Para que el idioma por defecto sea español
\usepackage[spanish]{babel}
%% Para poder subrayar entornos especiales como las secciones
\usepackage{ulem}

%% Texto matematico y simbolos especiales
\usepackage{amsmath}    % Paquete para mates
\usepackage{amsfonts}   % Paquete para mates
\usepackage{amssymb}    % Paquete para mates
\usepackage{stmaryrd}   % Paquete para mates
\usepackage{latexsym}   % Paquete para mates

%% Paquete para incluir imágenes y ruta de la carpeta de las imágenes
\usepackage{graphicx}
\graphicspath{{./fotos/}}

%% Paquete para tener hipervínculos y referencias cruzadas
\usepackage[colorlinks=true]{hyperref}
\hypersetup{
	urlcolor=red,
	linkcolor=blue,
}

%% Paquete para incluir código con coloreado sintáctico
\usepackage{listings}
\lstdefinelanguage{C+-}
{
  keywords={
    if, els, guail, breic, continue,
    suich, queis, difolt,
    cein, ceaut,
    return
    },
  keywordstyle=\color{blue},
  emph={int, bul, func, tru, fols, estrut},
  emphstyle=\color{purple},
  commentstyle=\color{green},
  stringstyle=\color{red},
  sensitive=true,
  morecomment=[l]{//},
  moredelim=**[is][\color{green}]{@}{@},
}

\lstdefinestyle{customcode} {
    literate=*{{/*}{\char`/*}}{2} {*{/}{\char`*/}}{2} {*/}{{{\color{commentgreen}*/}}}3,
}

\lstset{
    language=C+-,
    basicstyle=\ttfamily\small,
    keywordstyle=\color{blue},
    commentstyle=\color{green},
    stringstyle=\color{red},
    numbers=left,
    numberstyle=\tiny\color{gray},
    breaklines=true,
    frame=shadowbox,
    rulesepcolor=\color{black},
    backgroundcolor=\color{white},
    tabsize=2,
    gobble=12,
    linewidth=0.65\linewidth,
    float=h
}


%% Definicion de operadores especiales para simplificar la escritura matematica
\DeclareMathOperator{\dom}{dom}
\DeclareMathOperator{\img}{img}
\DeclareMathOperator{\rot}{rot}
\DeclareMathOperator{\divg}{div}
\newcommand{\dif}[1]{\ d#1}

%% Paquete e instrucciones para la generacion de los dibujos
\usepackage{pgfplots}
\pgfplotsset{compat=1.17}
\usepackage{tkz-fct}
\usepackage{pstricks}
\usepackage{pstcol} 
\usepackage{pst-node}
\usepackage{pst-plot}

%% Paquetes extra
\usepackage{centernot}  % Paquete para tachar cosas
\usepackage{appendix}   % Paquete para apéndice
\usepackage{verbatim}   % Paquete para comentar bloques de código de LaTeX
\usepackage{multicol}

\begin{document}
\maketitle
\tableofcontents

\section{Especificaciones técnicas del lenguaje}
\subsection{Identificadores y ámbitos de definición}
El lenguaje posee las siguientes características:
\begin{itemize}
    \item \textbf{Declaración de variables}: se pueden declarar variables sencillas de los tipos definidos y variables \textit{array} de estos tipos, de cualquier dimensión. Los nombres de los identificadores han de ser expresiones alfanuméricas que no comiencen por números y que posiblemente tengan el caracter ``\_''.
    \item \textbf{Bloques anidados}: se permiten las anidaciones en condicionales, bucles, funciones, etc. Si dos variables tienen el mismo nombre, la más profunda (en la anidación) tapa a la más externa.
    \item \textbf{Funciones}: se permite la creación de funciones y la declaración implícita dentro de otras. El paso por valor y por referencia de cualquier tipo a las funciones está garantizado.
    \item \textbf{Punteros}: para cada tipo se puede declarar un puntero a una variable de ese tipo, mediante la asignación de su dirección de memoria a la variable puntero.
    \item \textbf{Registros y clases}: se incluyen dos tipos adicionales: los registros como ``saco de datos'' ---sin métodos--- y las clases, tanto con datos como con métodos de función.
    \item \textbf{Declaración de constantes}: se incluye la posibilidad de declarar constantes por parte del usuario.
\end{itemize}

\subsection{Tipos}
La declaración de tipos ha de hacerse de manera explícita y de forma previa al lugar donde se emplee el identificador, es decir, que para poder usar una variable tengo que haberla declarado antes.

\subsubsection{Enteros y booleanos}
Los tipos básicos del lenguaje son los enteros y los booleanos. La sintaxis de declaración de estos tipos es la siguiente:
\begin{itemize}
    \item \textbf{Enteros}: \texttt{\color{blue} int var;}
    
    Entre las operaciones habilitadas para el tipo:
    \begin{itemize}
        \item Suma: \texttt{\color{blue} a + b}
        \item Resta: \texttt{\color{blue} a - b}
        \item Multiplicación: \texttt{\color{blue} a * b}
        \item División: \texttt{\color{blue} a / b}
        \item Potencia: \texttt{\color{blue} a\^{}b}
        \item Paréntesis: \texttt{\color{blue} ()}
        \item Menor: \texttt{\color{blue} a \textless{} b}
        \item Mayor: \texttt{\color{blue} a \textgreater{} b}
        \item Igual: \texttt{\color{blue} a = b}
        \item Menor o igual: \texttt{\color{blue} a \textless{}= b}
        \item Mayor o igual: \texttt{\color{blue} a \textgreater{}= b}
        \item Distinto: \texttt{\color{blue} a != b}
    \end{itemize}
    \item \textbf{Booleanos}: \texttt{\color{blue} bul var;}
    
    Entre las operaciones habilitadas para el tipo:
    \begin{itemize}
        \item \textit{Y} lógico: \texttt{\color{blue} a an b}
        \item \textit{O} lógico: \texttt{\color{blue} a or b}
        \item \textit{No} lógico: \texttt{\color{blue} !a}
    \end{itemize}
    Además, las palabras reservadas para indicar el valor verdadero o falso son \texttt{\color{blue}tru} y \texttt{\color{blue}fols}.
\end{itemize}


\subsubsection{Clases y registros}
Como tipos adicionales hemos incluido los registros, las clases y las funciones. La sintaxis de declaración es la siguiente:
\begin{itemize}
    \item \textbf{Clases}: \texttt{\color{blue} clas var \{...\};}
    
    Entre las operaciones habilitadas para el tipo:
    \begin{itemize}
        \item Acceso a campos: \texttt{\color{blue} var.campo}
        \item Acceso a métodos: \texttt{\color{blue} var.metodo}
        \item Constructor: \texttt{\color{blue} var(args)}
    \end{itemize}
    \item \textbf{Registros}: \texttt{\color{blue} estrut var \{...\};}
    
    Entre las operaciones habilitadas para el tipo:
    \begin{itemize}
        \item Acceso a campos: \texttt{\color{blue} var.campo}
        \item Constructor: \texttt{\color{blue} var(args)}
    \end{itemize}
\end{itemize}


\subsubsection{Arrays}
Todos los tipos pueden formar un array multidimensional, la sintaxis de declaración es la siguiente:
\begin{itemize}
    \item \textbf{Array}: \texttt{\color{blue} Tipo[DIMENSION] var;}
    
    Entre las operaciones habilitadas para el tipo:
    \begin{itemize}
        \item Operador de acceso: \texttt{\color{blue} var[INDEX]}
    \end{itemize}
\end{itemize}



\subsubsection{Funciones}
Las funciones se han declarado también como un tipo para poder hacer expresiones lambda y pasar funciones como argumento. La sintaxis de declaración de una función es la siguiente:
\begin{itemize}
    \item \textbf{Funciones}: \texttt{\color{blue} func var(Tipo arg1, Tipo arg2, ...) : TipoRetorno \{...\};}
\end{itemize}
El paso de parámetros por defecto es por valor, pero puede cambiarse a por referencia añadiendo el caracter ``\&'' al final del tipo del argumento.

\subsubsection{Punteros}
Pueden declarase punteros a cualquiera de los tipos definidos. La sintaxis de declaración es:
\begin{itemize}
    \item \textbf{Puntero\footnote{Como nota especial, la declaración de un puntero a estructuras de tipo array, se haría como \texttt{Tipo[DIMENSION]~ var;}.}}: \texttt{\color{blue} Tipo\~{} var}
    
    Entre las operaciones habilitadas para el tipo:
    \begin{itemize}
        \item Asociación a memoria dinámica: \texttt{\color{blue} var := niu Tipo}
        \item Acceso al dato: \texttt{\color{blue} \~{}var}
    \end{itemize}
\end{itemize}

\subsubsection{Tipos definidos por el usuario y constantes}
Adicionalmente, permitimos la definición de tipos por parte del usuario a través de la palabra reservada:
\begin{itemize}
    \item \textbf{Definición de tipos de usuario}: \texttt{\color{blue} taipdef nombre expresion}.
\end{itemize}
Por último, la declaración de constantes se ejecuta a través de la expresión:
\begin{itemize}
    \item \textbf{Declaración de constantes}: \texttt{\color{blue} difain NOMBRE valor}
\end{itemize}

\subsection{Instrucciones del lenguaje}
A continuación se presenta el repertorio de instrucciones del lenguaje. Préstese atención a aquellas que terminan con el caracter ``;'' para delimitar su final:
\begin{itemize}
    \item \textbf{Instrucción de asignación}: \texttt{:=}
    \begin{center}
        \begin{minipage}{\linewidth}
            \begin{lstlisting}[linewidth=0.3\linewidth, gobble=16]
                int var := 3;
            \end{lstlisting}
        \end{minipage}
    \end{center}
    
    \item \textbf{Instrucciones condicionales}: \texttt{if-els}, \texttt{suich-queis}
    \begin{center}
        \begin{minipage}{\linewidth}
            \begin{lstlisting}[linewidth=0.3\linewidth, gobble=16]
                if (var > 3) {
                    ...
                }
                els {
                    ...
                }
            \end{lstlisting}
        \end{minipage}
    \end{center}
    \begin{center}
        \begin{minipage}{\linewidth}
            \begin{lstlisting}[linewidth=0.3\linewidth, gobble=16]
                suich (var) {
                    queis(val1):
                        ...
                        breic;
                    queis(val2):
                        ...
                        breic;
                    difolt:
                        ...
                        breic;
                }
            \end{lstlisting}
        \end{minipage}
    \end{center}

    \item \textbf{Instrucción de bucle}: \texttt{while}
    \begin{center}
        \begin{minipage}{\linewidth}
            \begin{lstlisting}[linewidth=0.3\linewidth, gobble=16]
                guail (var > 0) {
                    ...
                }
            \end{lstlisting}
        \end{minipage}
    \end{center}
    Se incluyen además las instrucciones \texttt{\color{blue} breic} y \texttt{\color{blue} continiu}.

    \item \textbf{Instrucciones de entrada salida}: \texttt{\color{blue} cein()} y \texttt{\color{blue} ceaut()}
    \begin{center}
        \begin{minipage}{\linewidth}
            \begin{lstlisting}[linewidth=0.3\linewidth, gobble=16]
                cein(var);
                ceaut(var);
            \end{lstlisting}
        \end{minipage}
    \end{center}
    
    \item \textbf{Instrucción de retorno de función}: \texttt{\color{blue} return}
    \begin{center}
        \begin{minipage}{\linewidth}
            \begin{lstlisting}[linewidth=0.3\linewidth, gobble=16]
                func foo(Tipo arg) : TipoRetorno {
                    TipoRetorno var;
                    ...
                    return var;
                }
            \end{lstlisting}
        \end{minipage}
    \end{center}

\end{itemize}

\subsection{Elementos de la estructura del código}
De forma idéntica a C++ el código se empezará a ejecutar a partir de la funcíon mein.

De forma también idéntica, se incluye la posibilidad de escribir comentarios en el código, esto es, líneas que en el momento de compilación serán ignoradas. Se permiten tanto comentarios de una línea, encabezados por ``//'' y terminados en un salto de línea, como comentarios multilínea, encabezados por ``/*'' y terminados con ``*/''.

Se muestra a continuación un ejemplo de código que incluye estos elementos:

\begin{center}
    \begin{minipage}{\linewidth}
        \begin{lstlisting}[linewidth=0.7\linewidth, gobble=16]
                func mein() : int {
                    // La ejecucion del programa empieza aqui
                    int var := 3;
                    @/*@
                    @La llamada a la funcion siguiente calcula el @
                    @factorial de la variable var.@
                    @Almacenamos el valor en la variable res para@
                    @mostrarla por pantalla a continuacion.@
                    @*/@
                    int res := factorial(var);
                    ceaut(res);
                    return 0;
                }
        \end{lstlisting}
    \end{minipage}
\end{center}

\newpage
\appendix

\section{Ejemplos de programas habituales}

\lstinputlisting[language=C+-, linewidth=\linewidth, caption={Ejemplo de programa típico en $C+-$.}]{../codes/mainExample.cpm}
\lstinputlisting[language=C+-, linewidth=\linewidth, caption={Ejemplo de programa iterativo en $C+-$.}]{../codes/iterativeExample.cpm}
\lstinputlisting[language=C+-, linewidth=\linewidth, caption={Ejemplo de arrays multidimensionales en $C+-$.}]{../codes/matrixExample.cpm}
\lstinputlisting[language=C+-, linewidth=\linewidth, caption={Ejemplo de programa recursivo y sentencia \texttt{if-els} en $C+-$.}]{../codes/conditionalExample.cpm}
\lstinputlisting[language=C+-, linewidth=\linewidth, caption={Ejemplo de uso de registros y sentencia \texttt{suich} en $C+-$.}]{../codes/structExample.cpm}

\end{document}
